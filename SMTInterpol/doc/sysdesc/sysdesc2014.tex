\documentclass{article}
\usepackage[english]{babel}
\usepackage{xspace}
\usepackage{hyperref}

\newcommand\SI{SMTInterpol\xspace}
\newcommand\version{2.1-96-gade4dc0-comp\xspace}

\title{\SI\\{\large Version \version}}

\author{J\"urgen Christ, Jochen Hoenicke\\
  University of Freiburg\\
  \texttt{\{christj,hoenicke\}@informatik.uni-freiburg.de}}

\begin{document}
\maketitle
\section*{Description}
\SI is an SMT solver written in Java and available under LGPL v3.  It supports
the quantifier-free combination of the theories of uninterpreted functions,
linear arithmetic over integers and reals, and arrays.  Furthermore it can
produce models, proofs, unsatisfiable cores, and interpolants.  The solver
reads input in SMTLIB format.  It includes parsers for DIMACS, AIGER, and
SMTLIB version 1.2 and 2.

The solver uses variants of standard algorithms for CNF
conversion~\cite{DBLP:journals/jsc/PlaistedG86}, congruence
closure~\cite{DBLP:conf/rta/NieuwenhuisO05}, simplex~\cite{DBLP:conf/cav/DutertreM06} and
branch-and-cut~\cite{DBLP:conf/cav/DilligDA09}.
The array decision procedure is based on \emph{weak equivalences} and will be
presented at the SMT workshop 2014.
%%   A new array decision
%% procedure based on \emph{weak equivalences} has been added to \SI for this years
%% competition.  Details about this procedure will be presented at the SMT
%% workshop.
Theory combination is performed based on partial models produced by the theory
solvers~\cite{DBLP:journals/entcs/MouraB08}.

The main focus of \SI is the application track where the incremental usage of
the solver is required.  This track simulates the typical application of \SI
where a user asks multiple queries.  Unfortunately, the interpolation
engine~\cite{DBLP:conf/tacas/ChristHN13} of \SI which is the focus of the
development team of \SI is not tested during SMT-COMP.

\section*{Competition Version}
The version submitted to the SMT-COMP 2014 is a preliminary release of version
2.2.  This release will include quantifier-free interpolation for the theory
of arrays which is still missing in the current solver.

Further information about \SI can be found at
\begin{center}
  \url{http://ultimate.informatik.uni-freiburg.de/smtinterpol/}
\end{center}
The sources are available via GitHub
\begin{center}
  \url{https://github.com/juergenchrist/smtinterpol}
\end{center}

Magic Number: $3\,154\,861\,623$

\bibliography{sysdec}
\bibliographystyle{alpha}
\end{document}
