\documentclass{article}
\usepackage[english]{babel}
\usepackage{xspace}
\usepackage{hyperref}

\newcommand\SI{SMTInterpol\xspace}
\newcommand\version{2.0pre\xspace}

\title{\SI\\{\large Version \version}}

\author{J\"urgen Christ, Jochen Hoenicke\\
  University of Freiburg\\
  \texttt{\{christj,hoenicke\}@informatik.uni-freiburg.de}}
\date{July 8, 2011}

\begin{document}
\maketitle
\section*{Description}
\SI is a proof-producing SMT-solver written in Java.  It stores resolution proof trees to
compute interpolants~\cite{mcmillan05itp} used by different model checking
tools~\cite{HHP09,HHP10}.

The solver reads input in SMTLIB format.  It includes a parser for version
1.2, and for the current version.  All required and some optional commands of
the SMTLIB standard are supported.

All formulas are stored in a central term repository.  The repository
type-checks the formulas and does some simple boolean optimizations.  Asserted
formulas are converted to CNF using Plaisted--Greenbaum
encoding~\cite{DBLP:journals/jsc/PlaistedG86}.  The core of the solver is a
CDCL engine that is connected to multiple theories.  The engine uses these
theories during constraint propagation, backtracking, and consistency
checking.

For uninterpreted functions and predicates, we use a theory solver based on
the congruence closure algorithm.  An extension to arrays and quantifiers via
e-matching is under development.  For linear arithmetic, we use a theory
solver based on the Simplex algorithm~\cite{DBLP:conf/cav/DutertreM06}.  It
always computes the strongest bounds that can be derived for a variable and
uses them during satisfiability checks.  If a conflict cannot be explained
using known literals, the solver derives new literals and uses them in
conflict explanation.  Disequalities are resolved if they can be used to
strengthen a bound.  Otherwise, they are delayed until final checks.  The
solver supports integer arithmetic using a variant of the cuts from proof
technique~\cite{DBLP:conf/cav/DilligDA09} together with a branch-and-bound
engine.

\SI uses a variant of model-based theory
combination~\cite{DBLP:journals/entcs/MouraB08}.  The linear arithmetic solver
does not propagate equalities between shared variables but introduces them as
decision points.  The model mutation algorithm resolves disequalities
and tries to create as many distinct equivalence classes as possible.

\section*{Competition Version}
The version submitted to the SMT-COMP 2011 is a preliminary version.  Some
major features of the new SMTLIB standard are still under development or
test.  Future versions will include extensions to quantifiers, arrays, models,
and a more flexible interpolation scheme.

This version is yet to be released, but the previous version of the solver can
be downloaded from
\begin{center}
  \url{http://swt.informatik.uni-freiburg.de/research/tools/smtinterpol}
\end{center}

Magic Number: 6649887

\bibliography{sysdec}
\bibliographystyle{alpha}
\end{document}
